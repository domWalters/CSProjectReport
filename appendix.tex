\section{Appendix}
\subsection{Relevant Intrinio Field Meanings} \label{intrinioFields}
\subsubsection*{adjbasiceps}
Basic earnings per share ``is a rough measurement of the amount of a company's profit that can be allocated to one share of its stock.''\cite{basicEarnings} The adjusted form then accounts for specific anomalies in the data. The formula to calculate basic EPS is: \newline 

Basic EPS = $ \frac{Net \, \, income – Preferred \, \, dividends}{Weighted \, \, average \, \, number \, \, of \, \, common \, \, shares \, \, outstanding}$

\subsubsection*{adjweightedavebasicdilutedsharesos} \label{adjweightedavebasicdilutedsharesos}
Adjusted Weighted Average Basic \& Diluted Shares Outstanding is a metric that ``incorporates any changes in the amount of outstanding shares (both Basic and Diluted) over a reporting period''. It is used to more accurately measure EPS for companies where the number of available stocks can drastically change on a regular basis. \cite{weightedOutstandingShares}

\subsubsection*{adjweightedavebasicsharesos}
Adjusted Weighted Average Basic Shares Outstanding is the same as \ref{adjweightedavebasicdilutedsharesos} except now only the Basic Shares are considered.

\subsubsection*{bookvaluepershare}
Book value per share is a ``method to calculate the per-share value of a company based on common shareholders' equity in the company''. \newline

``Should the company dissolve, the book value per common share indicates the dollar value remaining for common shareholders after all assets are liquidated and all debtors are paid.'' \cite{bookValuePerShare}

\subsubsection*{croic}
Cash Return On Invested Capital ``measures how much cash a company generates based on each dollar it invests into its operations.'' \newline

``The higher the CROIC, the better and a CROIC above 10\% is usually regarded as good.''\cite{croic}

\subsubsection*{debttonopat} \label{debttonopat}
Debt to NOPAT = $\frac{Debt}{Net \, \, Operating \, \, Profit \, \, After \, \, Tax}$. \newline

Quite clearly it would be preferable that this number shouldn't be too high, but equally it shouldn't be low as otherwise the company isn't effectively leveraging their borrowing potential.\cite{intrinioDataTags}

\subsubsection*{debttototalcapital}
Debt to Total Capital = $\frac{Debt}{Total \, \, Capital}$. \newline

Similarly to \ref{debttonopat}, the debt-to-capital ratio is a measurement of a company's financial leverage.\cite{intrinioDataTags}

\subsubsection*{dividendyield}
Dividend Yield is a financial ratio that indicates how much a company pays out in dividends to it's shareholders each year relative to its share price.\cite{intrinioDataTags}

\subsubsection*{ebit}
EBIT, or Earnings before Interest and Taxes, ``measures the profit a company generates from its operations, making it synonymous with operating profit.'' It ignores tax and interest expenses to ``focus solely on a company's ability to generate earnings from operations''.\cite{intrinioDataTags}

\subsubsection*{ebitda}
Earnings before Interest, Taxes, Depreciation, and Amortization is the same type of metric as EBIT, except with a few more areas of expense ignored.

\subsubsection*{ebitgrowth}
EBIT Growth is the year-on-year percent change in EBit.

\subsubsection*{ebitlesscapextointerestex}
EBIT Less CapEx to Interest Expense = $\frac{EBIT - CapEx}{Total \,\, Interest \,\, Expense}$.\cite{intrinioDataTags}

\subsubsection*{ebitqoqgrowth}
EBIT QoQ Growth is the quarter-on-quarter percent change in EBit.

\subsubsection*{ebittointerestex}
EBIT to Interest Expense is a ``measurement of how much a company is earning over its interest payments (or how easily a company can pay interest on outstanding debt). A ratio of three means that a company is making three times its interest payment expense.''\cite{intrinioDataTags}

\subsubsection*{evtoebitda} \label{evtoebitda}
Enterprise Value to EBITDA ``allows investors to compare the value of a company, debt included, to the companys cash earnings less its noncash expenses. It is ideal for analysts and potential investors looking to compare companies within the same industry. Typically, EV/EBITDA values below 10 are seen as healthy.''\cite{intrinioDataTags}

\subsubsection*{evtofcff}
Enterprise Value to Free Cash Flow ``compares the total valuation of the company with its ability to generate cashflow. ... The lower the ratio ... , the faster a company can pay back the cost of its acquisition or generate cash to reinvest in its business.''\cite{evtofcf}

\subsubsection*{evtonopat}
Enterprise Value to NOPAT is a similar metric to \ref{evtoebitda} except now using NOPAT as the measure of income.

\subsubsection*{evtoocf}
Enterprise value to Operating cash flow is ``the ratio of the entire economic value of a company to the cash it produces. ... In other words, how long does it take the company to pay for itself?''\cite{intrinioDataTags}

\subsubsection*{faturnover}
Fixed-asset turnover is used to measure operating performance. \newline

FATurnover = $\frac{Total \,\, Revenue}{Net \,\, Premises \,\, and \,\, Equipment}$ \newline

``In general, a higher fixed-asset turnover indicates that a company has more effectively utilized investment in fixed assets to generate revenue.''\cite{intrinioDataTags}

\subsubsection*{finleverage}
Financial Leverage is ``the degree to which a company uses fixed-income securities, such as debt and preferred equity. With a high degree of financial leverage come high interest payments. As a result, the bottom-line earnings per share is negatively affected by interest payments.''\cite{intrinioDataTags}

\subsubsection*{investedcapital}
Invested Capital = $(totalassets - (totalliabilities - shorttermdebt - longtermdebt - capitalleaseobligations))$. It is ``the total amount of money raised by a company by issuing securities to shareholders and bondholders.''\cite{intrinioDataTags}

\subsubsection*{investedcapitalqoqgrowth}
Invested Capital QoQ Growth is the quarter-on-quarter percent change in Invested Capital.

\subsubsection*{ltdebtandcapleases}
Total Long Term Debt ``consists of loans and financial obligations lasting over one year.''\cite{intrinioDataTags}

\subsubsection*{marketcap}
Market Capitalization ``refers to the total dollar market value of a company's outstanding shares.'' This is used as to determine a company's size. \newline

marketcap = $adj\_close\_price \cdot weightedavedilutedsharesos$.\cite{intrinioDataTags}

\subsubsection*{netnonopex}
Net Non-Operating Expense is ``an expense incurred by a business that's unrelated to its core operations. The most common types of non-operating expenses relate to depreciation, amortization, interest charges or other costs of borrowing.''\cite{intrinioDataTags}

\subsubsection*{nopatqoqgrowth}
NOPAT QoQ Growth is the quarter-on-quarter percent change in NOPAT.

\subsubsection*{ocftointerestex}
Operating Cash Flow to Interest Expense = $\frac{Net \,\, Cash \,\, from \,\, Operating \,\, Activities}{Total \,\, Interest \,\, Expense}$.\cite{intrinioDataTags}

\subsubsection*{pricetorevenue}
Price to Revenue is ``a valuation ratio that compares a company's stock price to its revenues. The price-to-revenue ratio is an indicator of the value placed on each dollar of a company's revenues. ... A low ratio may indicate possible undervaluation, while a ratio that is significantly above the average may suggest overvaluation.''\cite{intrinioDataTags}

\subsubsection*{sgaextorevenue}
Selling, General, \& Administrative Expenses to Revenue is a measure of profitability. The lower this figure, the more money the company made per unit expense. \cite{sgaextorevenue}

\subsubsection*{totalassets}
Total assets is ``the sum of all current and noncurrent assets that a company owns.''\cite{intrinioDataTags}

\subsubsection*{totalcapital}
Total capital refers to ``the sum of long-term debt and total shareholder equity. This is a calculation that is traditionally used when determining a company's return on capital.''\cite{intrinioDataTags}

\subsubsection*{totalequityandnoncontrollinginterests}
Total equity and non controlling interests are ``the sum of preferred stock, common equity and non controlling interest.''\cite{intrinioDataTags} This figure represents the total amount of money invested by all three kinds of shareholder.

\subsubsection*{totalliabilities}
Total liabilities refer to ``the aggregate of all debts for which an individual or company is liable.'' \cite{totalLiabilities}

\begin{figure}[p]
    \subsection{Tables}
    \subsubsection{Varying Generations} \label{varyGen}
    {\centering
    \pgfplotstabletypeset[
      columns/run/.style={column name=Number of Runs},
      columns/10/.style={column name=10 Gens},
      columns/15/.style={column name=15 Gens},
      columns/20/.style={column name=20 Gens},
      columns/25/.style={column name=25 Gens},
    ]{tables/generations-default-full.txt}
    \pgfplotstabletypeset[
      columns/gen/.style={column name=Generations},
      columns/mean/.style={column name=Mean(\%)},
      columns/median/.style={column name=Median(\%)},
    ]{tables/generations-default.txt}
    \\}
    \subsubsection{Varying Iterations} \label{varyIter}
    {\centering
    \pgfplotstabletypeset[
      columns/run/.style={column name=Number of Runs},
      columns/2/.style={column name=2 Iters},
      columns/3/.style={column name=3 Iters},
      columns/4/.style={column name=4 Iters},
      columns/5/.style={column name=5 Iters},
    ]{tables/iterations-default-full.txt}
    \pgfplotstabletypeset[
      columns/iter/.style={column name=Iterations},
      columns/mean/.style={column name=Mean(\%)},
      columns/median/.style={column name=Median(\%)},
    ]{tables/iterations-default.txt}
    \\}
\end{figure}

\begin{figure}[p]
    \subsubsection{Varying Population Size} \label{varyLambda}
    {\centering
    \pgfplotstabletypeset[
      columns/run/.style={column name=Number of Runs},
      columns/50/.style={column name=50 Pop},
      columns/100/.style={column name=100 Pop},
      columns/150/.style={column name=150 Pop},
      columns/200/.style={column name=200 Pop},
    ]{tables/lambda-default-full.txt}
    \pgfplotstabletypeset[
      columns/lambda/.style={column name=Population Size},
      columns/mean/.style={column name=Mean(\%)},
      columns/median/.style={column name=Median(\%)},
    ]{tables/lambda-default.txt}
    \\}
    \subsubsection{Varying Percentile Gap} \label{varyPerc}
    {\centering
    \pgfplotstabletypeset[
      columns/run/.style={column name=Number of Runs},
      columns/1/.style={column name=Gaps of 1},
      columns/2/.style={column name=Gaps of 2},
      columns/4/.style={column name=Gaps of 4},
      columns/5/.style={column name=Gaps of 5},
      columns/10/.style={column name=Gaps of 10},
    ]{tables/percentiles-default-full.txt}
    \pgfplotstabletypeset[
      columns/percentiles/.style={column name=Percentile Gap},
      columns/mean/.style={column name=Mean(\%)},
      columns/median/.style={column name=Median(\%)},
    ]{tables/percentiles-default.txt}
    \\}
\end{figure}