\section{Project Layout}
The project can be split into three core sections.

\subsection{The Simulation Game} \label{simGameProp}
In order to train a genetic algorithm over historical data, we have to create some form of “payoff” for each of the proposed screening strategies in the population, that is representative of how much profit that strategy would create when choosing stocks to buy/sell in the real economy. \newline

My proposed way to achieve this is to play a simulated stock trading “game”. For each prospective screening rule, run through as many quarters of historical data as possible. For each quarter, the screening strategy buys all stocks that satisfy it from the quarter. At the end of the game, the net worth of the strategy is calculated and a payoff is assigned based on the profit attained. \newline

This payoff is then used to perform the selection function required for the genetic algorithm, as well as to contextualise the created screening strategies and compare to the returns of the S\&P 500.

\subsection{The Evolutionary Algorithm} \label{evoAlgoProp}
The pseudo-code for a generic evolutionary algorithm is as follows:
\begin{enumerate}
    \item \bf Initialisation \rm - Generate initial population $gen_{0}$ \vspace{-2mm}
    \item For g from \{1, 2, .., G\}
    \begin{enumerate}
        \item Create blank population $gen_{g}$
        \item For \_n from \{0, 1, .., size of $gen_{g-1}$ - 1\}
        \begin{enumerate}
            \item \bf Selection \rm - Select two members of $gen_{g-1}$, biasing to members with higher \bf Fitness Function \rm values
            \item \bf Crossover \rm - Create a new member by crossover of the two selected members
            \item \bf Mutation \rm - Mutate the new member
            \item Add the new member to $gen_{g}$
        \end{enumerate}
    \end{enumerate} \vspace{-2mm}
    \item Return $gen_{G}$
\end{enumerate}

I intend to begin with this generic framework, and then adapt my algorithm as needed. However, not all of this standard version is straight forward and easy to implement within the search space I am considering. Specifically:
\begin{itemize}
    \item \bf Generation of Initial Population \rm 
    \item \bf Computation of the Fitness Function \rm
    \item \bf The Crossover Function \rm
\end{itemize}

This section of the system will require experimentation, as the initialisation and crossover methods will need to be specially designed for this search space. I anticipate that the crossover function is going to be the hardest element to construct. \newline

It has to allow a combination of two screening rules of a hundred variables each (at a minimum), all representing different aspects of a stock (with different scales and potentially wildly different values), in such a way that the result will on average be a better screening strategy. This does not seem like a trivial task.

\subsection{Testing and Experimentation}
Once the algorithm is established and providing valid results, a series of evaluations should be run to determine the quality of the screening strategies provided. This is likely to be a highly iterative process: a test is run, an improvement suggested, the test is rerun with the improvement. \newline

Additionally, more complicated evolutionary methods could be tested: niching and speciation strategies, elitism, co-evolution, and multi-objective optimisation to name a few. It is also at this stage that the algorithm’s variables can be tuned to attain the best screening rule possible. \newline

Evaluation of the system is very open-ended in nature, only really limited by what can be achieved before the deadline.